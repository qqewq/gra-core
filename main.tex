\documentclass[11pt,a4paper]{article}

% Encoding and language
\usepackage[utf8]{inputenc}
\usepackage[T1]{fontenc}
\usepackage[english,russian]{babel}

% Math
\usepackage{amsmath, amssymb, amsthm}
\usepackage{bm}
\usepackage{geometry}
\geometry{margin=1in}

% Algorithms (for pseudocode)
\usepackage{algorithm}
\usepackage{algorithmic}

\title{GRA-Core: Multiverse Nullification and Fractal Agents\\
{\large Базовый стек GRA: мультиверсное обнуление и фрактальные агенты}}

\author{AAA AAA}

\date{February 2026}

% Theorem environments
\newtheorem{theorem}{Theorem}
\newtheorem{definition}{Definition}

\begin{document}

\maketitle

\begin{abstract}
\selectlanguage{english}
We present \textbf{GRA-Core}, a minimal implementation of the GRA Multiverse
Nullification architecture. The framework provides: (i) a multilevel foam
nullification functional over tensorized state spaces, (ii) a fractal
context engine for language and knowledge, and (iii) an agent runtime with
GRA-style cognitive reset. We outline the core mathematical ideas and
their differentiable realization in NumPy and JAX, accompanied by open-source
notebooks for experimentation.

\selectlanguage{russian}
В работе представлен \textbf{GRA-Core} — минимальная реализация архитектуры
мультиверсного обнуления GRA. Фреймворк включает: (i) многоуровневый
функционал обнуления «пены» в тензорных пространствах состояний, (ii)
фрактальный контекстный движок для языка и знаний и (iii) агентный
рантайм с GRA-подобным когнитивным сбросом. Описаны основные математические
идеи и их дифференцируемая реализация на NumPy и JAX, а также открытые
ноутбуки для экспериментов.
\end{abstract}

\section{Introduction / Введение}

\selectlanguage{english}
Large-scale learning systems accumulate internal contradictions across
layers, modules, and abstraction levels. This manifests as hallucinations,
instabilities, and brittle behaviour. The GRA Multiverse Nullification
principle treats these contradictions as ``foam'' between parallel
state components and proposes a recursive nullification process over
a multilevel hierarchy.

\selectlanguage{russian}
Крупномасштабные обучающиеся системы накапливают внутренние противоречия
между слоями, модулями и уровнями абстракции. Это проявляется в виде
галлюцинаций, нестабильности и хрупкого поведения. Принцип мультиверсного
обнуления GRA трактует эти противоречия как «пену» между параллельными
компонентами состояний и предлагает рекурсивный процесс обнуления
на многоуровневой иерархии.

\section{Multiverse State Space / Пространство состояний мультиверса}

\selectlanguage{english}
Let the multilevel index be
\[
\bm{a} = (a_0, a_1, \dots, a_k),
\]
where $a_0$ indexes domains, $a_1$ meta-systems, and so on up to
level $k$. For each level $l$ we define
\[
\mathcal{H}^{(l)} = \bigotimes_{\bm{a}:\,\dim(\bm{a})=l} \mathcal{H}^{(\bm{a})},
\quad
\mathcal{H}_{\text{multiverse}} = \bigotimes_{l=0}^K \mathcal{H}^{(l)}.
\]

\selectlanguage{russian}
Пусть мультииндекс имеет вид
\[
\bm{a} = (a_0, a_1, \dots, a_k),
\]
где $a_0$ индексирует домены, $a_1$ — мета-системы и т.д. до уровня $k$.
Для каждого уровня $l$ определим
\[
\mathcal{H}^{(l)} = \bigotimes_{\bm{a}:\,\dim(\bm{a})=l} \mathcal{H}^{(\bm{a})},
\quad
\mathcal{H}_{\text{multiverse}} = \bigotimes_{l=0}^K \mathcal{H}^{(l)}.
\]

\section{Foam Functional / Функционал «пены»}

\selectlanguage{english}
For level $l$ and goal $G_l$ we introduce a projector
$\mathcal{P}_{G_l}$ onto the solution subspace. The level-$l$ foam for
a family of states $\{\Psi^{(\bm{a})}\}$ is
\[
\Phi^{(l)}(\Psi^{(l)}, G_l) =
\sum_{\bm{a} \neq \bm{b} \atop \dim(\bm{a})=\dim(\bm{b})=l}
\bigl|\langle \Psi^{(\bm{a})} \vert \mathcal{P}_{G_l} \vert
\Psi^{(\bm{b})} \rangle\bigr|^2.
\]

\selectlanguage{russian}
Для уровня $l$ и цели $G_l$ вводится проектор
$\mathcal{P}_{G_l}$ на подпространство решений. Пена уровня $l$ для
набора состояний $\{\Psi^{(\bm{a})}\}$ задаётся как
\[
\Phi^{(l)}(\Psi^{(l)}, G_l) =
\sum_{\bm{a} \neq \bm{b} \atop \dim(\bm{a})=\dim(\bm{b})=l}
\bigl|\langle \Psi^{(\bm{a})} \vert \mathcal{P}_{G_l} \vert
\Psi^{(\bm{b})} \rangle\bigr|^2.
\]

\section{Nullification Principle / Принцип обнуления}

\begin{definition}[Multiverse-consistent state / Мультиверсно-согласованное состояние]
\selectlanguage{english}
A multiverse state $\Psi \in \mathcal{H}_{\text{multiverse}}$ is called
\emph{multiverse-consistent} if for all levels $l$ the foam functional
vanishes:
\[
\Phi^{(l)}(\Psi^{(l)}, G_l) = 0.
\]

\selectlanguage{russian}
Состояние мультивселенной $\Psi \in \mathcal{H}_{\text{multiverse}}$
называется \emph{мультиверсно-согласованным}, если для всех уровней $l$
функционал пены обращается в ноль:
\[
\Phi^{(l)}(\Psi^{(l)}, G_l) = 0.
\]
\end{definition}

\begin{theorem}[Multiverse Nullification / Теорема мультиверсного обнуления]
\selectlanguage{english}
Assume that for each level $l$ and goal $G_l$ the projector
$\mathcal{P}_{G_l}$ is self-adjoint and idempotent, and that the
optimization dynamics performs a gradient-based descent on the total
foam
\[
J(\Psi) = \sum_{l=0}^K \Lambda_l \,\Phi^{(l)}(\Psi^{(l)}, G_l),\quad
\Lambda_l > 0.
\]
Then every limit point of the trajectory $\Psi_t$ under this descent
is multiverse-consistent, i.e.\ satisfies $\Phi^{(l)}(\Psi^{(l)}, G_l) = 0$
for all $l$.

\selectlanguage{russian}
Пусть для каждого уровня $l$ и цели $G_l$ проектор
$\mathcal{P}_{G_l}$ самосопряжён и идемпотентен, а динамика оптимизации
представляет собой градиентный спуск по суммарной пене
\[
J(\Psi) = \sum_{l=0}^K \Lambda_l \,\Phi^{(l)}(\Psi^{(l)}, G_l),\quad
\Lambda_l > 0.
\]
Тогда любая предельная точка траектории $\Psi_t$ при таком спуске
является мультиверсно-согласованной, то есть удовлетворяет
$\Phi^{(l)}(\Psi^{(l)}, G_l) = 0$ для всех $l$.
\end{theorem}

\noindent
\selectlanguage{english}
\textbf{Sketch of proof.}
The functional $J$ is non-negative and, under standard smoothness
assumptions on the parametrization of $\Psi$, monotonically decreases
along the gradient flow. Any limit point must therefore lie in the
set of critical points of $J$. The structure of $\Phi^{(l)}$ implies
that the only global minima are configurations in which all pairwise
interference terms vanish, which is equivalent to $\Phi^{(l)}=0$ for
all levels.

\selectlanguage{russian}
\textbf{Набросок доказательства.}
Функционал $J$ неотрицателен и при стандартных предположениях о
гладкости параметризации $\Psi$ монотонно убывает вдоль градиентного
потока. Любая предельная точка должна лежать в множестве критических
точек $J$. Структура $\Phi^{(l)}$ означает, что глобальные минимумы
соответствуют конфигурациям, в которых все попарные интерференционные
члены обнуляются, что эквивалентно выполнению $\Phi^{(l)}=0$ для
всех уровней.

\section{Algorithm / Алгоритм}

\selectlanguage{english}
Algorithm \ref{alg:gra-null} gives a high-level pseudocode for the
GRA multiverse nullification process as implemented in the JAX layer
of \texttt{gra-core}.

\selectlanguage{russian}
Алгоритм \ref{alg:gra-null} задаёт укрупнённый псевдокод процесса
мультиверсного обнуления GRA в реализации JAX из \texttt{gra-core}.

\begin{algorithm}[h]
\caption{GRA Multiverse Nullification / Мультиверсное обнуление GRA}
\label{alg:gra-null}
\begin{algorithmic}[1]
\STATE \textbf{Input / Вход:}
\STATE \quad initial states $\Psi^{(l)}_0$ for all levels $l$
\STATE \quad projectors $\mathcal{P}_{G_l}$, weights $\Lambda_l$, step size $\eta$
\FOR{$t = 0, 1, 2, \dots, T$}
  \STATE Compute multilevel foam:
  \[
    J(\Psi_t) = \sum_l \Lambda_l \,\Phi^{(l)}(\Psi^{(l)}_t, G_l)
  \]
  \STATE Compute gradient $\nabla_{\Psi} J(\Psi_t)$
  \STATE Update states:
  \[
    \Psi_{t+1} = \Psi_t - \eta \,\nabla_{\Psi} J(\Psi_t)
  \]
  \IF{$J(\Psi_t) < \varepsilon$}
    \STATE \textbf{break}
  \ENDIF
\ENDFOR
\STATE \textbf{Output / Выход:} multiverse-consistent $\Psi_{T^\star}$
\end{algorithmic}
\end{algorithm}

\section{Case Studies / Кейсы}

\subsection{Toy Nullification (NumPy) / Игрушечное обнуление (NumPy)}

\selectlanguage{english}
The notebook \texttt{nullify\_toy\_model.ipynb} instantiates a
two-level multiverse with random four-dimensional states and a simple
homogeneous projector. The evolution of $\Phi^{(0)}$ and
$\Phi^{(1)}$ over iterations illustrates exponential-like decay of
foam as states become aligned with the projector subspaces.

\selectlanguage{russian}
В ноутбуке \texttt{nullify\_toy\_model.ipynb} задаётся двухуровневый
мультиверс со случайными четырёхмерными состояниями и простым
однородным проектором. Эволюция $\Phi^{(0)}$ и $\Phi^{(1)}$ по
итерациям демонстрирует почти экспоненциальное затухание пены при
приближении состояний к подпространствам проекторов.

\subsection{Differentiable JAX Layer / Дифференцируемый слой JAX}

\selectlanguage{english}
The JAX notebook \texttt{jax\_nullify\_toy.ipynb} treats the multilevel
foam $J$ as a differentiable regularizer. Randomly initialized states
are optimized by gradient descent, and the loss trajectory confirms
that $J$ can be minimized efficiently with standard JAX tooling and
integrated into larger models.

\selectlanguage{russian}
В JAX-ноутбуке \texttt{jax\_nullify\_toy.ipynb} многоуровневая пена
$J$ выступает в роли дифференцируемого регуляризатора. Случайно
инициализированные состояния оптимизируются градиентным спуском;
траектория потерь показывает, что $J$ эффективно минимизируется
стандартными средствами JAX и может быть встроен в крупные модели.

\subsection{Agents and Cognitive Reset / Агенты и когнитивный сброс}

\selectlanguage{english}
The notebook \texttt{gra\_agent\_loop\_free.ipynb} illustrates how
GRA-style cognitive reset prevents unbounded growth of internal
agent memory. A pair of simple counting agents accumulate internal
state over several steps, after which a global reset is applied,
forcing the system back into a low-entropy configuration.

\selectlanguage{russian}
В ноутбуке \texttt{gra\_agent\_loop\_free.ipynb} показано, как
GRA-подобный когнитивный сброс предотвращает неограниченный рост
внутренней памяти агентов. Пара простых счётчиков накапливают
состояние в течение нескольких шагов, после чего применяется
глобальный сброс, возвращающий систему в низкоэнтропийную
конфигурацию.

\section{Conclusion / Заключение}

\selectlanguage{english}
GRA-Core is intended as a reference implementation of multiverse
nullification ideas, suitable both for experimentation and for
integration into larger systems. The NumPy and JAX layers make the
foam functional accessible as a drop-in regularizer or stability prior
for large models, while the fractal context and agent runtime
illustrate how the same principle can organize language and behaviour.

\selectlanguage{russian}
GRA-Core задуман как эталонная реализация идей мультиверсного
обнуления, пригодная как для экспериментирования, так и для
интеграции в более крупные системы. NumPy- и JAX-слои делают
функционал пены удобным как plug-in регуляризатор или стабилизирующий
prior для больших моделей, а фрактальный контекст и агентный рантайм
показывают, как тот же принцип может структурировать язык и поведение.

% Optional bibliography can be added here.

\end{document}
